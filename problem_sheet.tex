\documentclass[11pt,a4paper]{article} 
\usepackage{graphicx} 
\usepackage{amssymb, amsmath} 
\usepackage[margin=1cm]{geometry}
\usepackage[colorlinks=true,urlcolor=blue]{hyperref}

\begin{document}
\title{Writing Theses in \LaTeX: Task Sheet}
\author{Chris Cooling}
\maketitle
\pagestyle{empty}
\thispagestyle{empty}

\section{Creating a document}
\begin{itemize}
\item Create a new project called in Overleaf ``Thesis''
\item Create a new report document
\item Select a fontsize (11pt)
\item Select the type of page you want to write on (A4)
\item Make the document double-sided
\item Set the margins such that they are symmetric and over 1cm
\item Test how it looks with some blindtext or other nonsense text
\item Verify page numbers appear on each page
\end{itemize}

\section{Line Spacing}
\begin{itemize}
\item Use the ``linespread'' command to make the spacing for document 1.5 or 2 times
\item Include a section of document with single line spacing
\item Include a footnote with single-lined spacing
\end{itemize}

\section{Title Page}
\begin{itemize}
\item Make a title page you find aesthetically pleasing
\item Ensure it follows all the requirements laid down by Imperial
\end{itemize}

\section{Thesis Preamble}
\begin{itemize}
\item Add an abstract
\item Add a statement of originality
\item Add a copyright statement
\item Use blindtext or other placeholder text for the content
\end{itemize}

\section{Contents and Appendices}
\begin{itemize}
\item Add some chapters, appendices sections and sub-sections to your document
\item Add some tables and figures to your document
\item Add a table of contents, list of figures and list of tables to your document
\item Add a chapter to your table of contents named ``Supplementary Material'' without creating a new chapter in your pdf
\item Add a new table to the list of tables in this new chapter without the table appearing in your document
\item Add a new figure to the list of figures in this new chapter without the figure appearing in your document
\end{itemize}

\section{Splitting Your Document}
\begin{itemize}
\item Put the content for each chapter and appendix in a different .tex file
\item Use the ``include'' command to include each document in your thesis.tex file
\item Experiment using the ``includeonly'' command to only compile some of your chapters
\end{itemize}

\section{Hyperlinks}
\begin{itemize}
\item Use the ``hyperref'' package to make your references dynamic hyperlinks
\item Experiment with different colours for your links and choose one you like
\end{itemize}

\section{Subfigures}
\begin{itemize}
\item Find some pictures you like online (or use some from your research)
\item Place them in yuor project (click upload in the top-left of the Overleaf interface)
\item Insert a subfigure into your document
\item Give each subfigure a caption and a label
\item Give the figure a caption and a label
\item Reference the figure and a subfigure in some text
\end{itemize}

\section{Graphics Path}
\begin{itemize}
\item Create a different directory for each chapter of your thesis
\item Move your figures into the relevant directory
\item Set the graphicspath so it includes each of these directories
\end{itemize}

\section{Custom Commands}
\begin{itemize}
\item Create a custom command for a phrase you expect to often in your thesis
\item Create a custom command for a mathematical expression. It should take at least one argument.
\item Use both custom commands in your document
\end{itemize}

\section{Bibliography Management}
\begin{itemize}
\item Create a .bib file
\begin{itemize}
\item If you use a reference management software, ask it to export your reference library as a .bib file
\item Otherwise, make up a few references by hand
\end{itemize}
\item Cite a few of your references in your text
\item Use ``citet', ``citep'' and, at least once,  cite multiple references in a single command
\item Choose a referencing style you like
\end{itemize}

\section{Acronyms}
\begin{itemize}
\item Define some acronyms relevant to your topic
\item Use each acronym at least twice each in some sample text
\end{itemize}

\section{Aligning Equations}
\begin{itemize}
\item Define an alignat environment with at least three equations
\item Use at least one alignment point each equation to cause them to line up
\end{itemize}

\end{document}